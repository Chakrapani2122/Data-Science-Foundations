\begin{table*}[t]
\centering
\caption{Top ranked linguistic and signal-based features across multiple feature selection algorithms.}
\label{tab:weka_feature_ranking}
\scriptsize
\resizebox{\textwidth}{!}{
\begin{tabular}{c p{3.2cm} p{6.2cm} p{7.2cm}}
\toprule
\textbf{Rank} & \textbf{Feature Name} & \textbf{Algorithms \& Positions (Rank)} & \textbf{Reason for Selection} \\
\midrule
1 & FFT mean &
Rank 1: InfoGain, OneR, SymmetricalUncert, Correlation &
This is the dominant feature, ranking \#1 in four different algorithms. \\

2 & FFT max &
Rank 2: InfoGain, OneR, SymmetricalUncert &
It is highly correlated with the target variable alongside the mean. \\

3 & Word diversity &
Rank 1: CfsSubset, GreedyStepWise &
Selected as the absolute \#1 best feature by subset evaluators, indicating unique information not found in other features. \\

4 & Total number of words &
Rank 2: Correlation, GainRatio &
A foundational metric ranking in the Top 3 across multiple algorithms. \\

5 & FFT stddev &
Rank 4: CfsSubset, GreedyStepWise, InfoGain, SymmetricalUncert &
Extremely consistent across almost all evaluators, indicating high robustness. \\

6 & Punctuation characters ratio &
Rank 4: Correlation, OneR, Relief &
Consistently appears in the Top 5 for rank-based evaluators. \\

7 & Frequency of single quotes &
Rank 5: InfoGain, Relief, SymmetricalUncert &
A stylistic marker with high information gain. \\

8 & Soundex homogeneity hist bin 4 &
Rank 1: ClassifierAttributeEval &
Selected as the most important feature by classifier-based evaluation. \\

9 & Frequency of forward slash &
Rank 1: GainRatio &
The most distinctive attribute for data splitting under GainRatio. \\

10 & Automated readability index &
Rank 2: ClassifierAttributeEval &
Heavily relied upon by classifier-based evaluation despite low rankings elsewhere. \\
\bottomrule
\end{tabular}
}
\end{table*}